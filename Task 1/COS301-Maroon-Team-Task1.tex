\documentclass[a4paper,10pt]{article}

\usepackage[margin=2cm]{geometry}
\usepackage{graphicx}
\usepackage{amsmath}
\usepackage{array}
\usepackage{hyperref}
\usepackage[all]{hypcap}
\usepackage{listings}
\lstdefinestyle{TerminalStyle}{
  language=bash,
  basicstyle=\small\sffamily,
  numbers=left,
  numberstyle=\tiny,
  numbersep=3pt,
  frame=tb,
  columns=fullflexible,
  linewidth=0.9\linewidth,
  xleftmargin=0.1\linewidth
}
\lstdefinestyle{HtmlStyle}{
  language=html,
  basicstyle=\small\sffamily,
  numbers=left,
  numberstyle=\tiny,
  numbersep=3pt,
  frame=tb,
  columns=fullflexible,
  linewidth=0.9\linewidth,
  xleftmargin=0.1\linewidth
}
\lstdefinestyle{OutputStyle}{
  language=html,
  basicstyle=\small\sffamily,
  frame=tb,
  columns=fullflexible,
  linewidth=0.9\linewidth,
  xleftmargin=0.1\linewidth
}

\setlength{\parindent}{0pt}
\setlength{\parskip}{1ex plus 0.5ex minus 0.2ex}
\title{\includegraphics[width=12cm]{Eeufeeslogo.jpg} \\
       Department of Computer Science \\
       University of Pretoria \\
       \vspace{0.5cm}
       Software Engineering\\
       COS301 MiniProject \\
       \vspace{0.5cm}
       \begin{large} \textbf{Team Maroon}\\ NavUP\end{large}}

\date{} 
\author{Bondjobo, Jocelyn (J) 	13232852 		\\
		Mweshi, George (G)		16394713		\\
		Letsoalo, Joseph (J)	15043844		\\
		Setoaba, Phuti (P)		13032616		\\
		Trivella, Camron (C)	14070970		\\
		Coetzer, Albert (A)		15244882		\\
}

\begin{document}
\maketitle
\thispagestyle{empty}
\clearpage

\newpage
\pagenumbering{roman}
\thispagestyle{empty}
\tableofcontents
\clearpage

\newpage

\pagenumbering{arabic}

\section{Introduction}

	\subsection{Purpose} 	\subsection{Scope} 
	\subsection{Overview} 

\section{Overall Description}

	\subsection{Product Perspective}
	
		\subsubsection{System Interfaces}
			Each component and system/subsystem in the overall program will need its own interface between its own\\				software and the bigger program. For instance, the crowd sourcing routines must be able to record and report a\\			density map to the main branch of the program that must then be able to be accessed and utilized by an analyzer\\			which must then return it to the main branch to be retrieved as needed for navigation by the user.\\
		\subsubsection{User Interfaces}
			The user interface must consist of an easily understandable and navigable GUI, simple enough for even the\\				most inexperienced to use. The different options and account information must be visible and responsive and the\\			navigation component must be clear and accurate. The user should intuitively know how to get from any point to\\			any other point in the program.\\
		\subsubsection{Hardware Interfaces}
			The hardware interface must allow the program to access and utilize systems on the device on which the program\\			is running, specifically the GPS system if the program uses it, the wireless network adapter to check for Wi-Fi\\			routers and signal strength, the screen to display information and the device used to make selections. The\\				hardware interface will be the backbone of the program, allowing directions and instructions to be communicated\\			to the user.\\
		\subsubsection{Software Interfaces}
			The routines by which the program will be able to call, run and interpret other software on the device being\\				used to run the program. The routines will, for example, utilize the software responsible for managing the\\ 				connection between the device and the Wi-Fi routers. This will co-exist with the hardware interface to make use\\			of the hardware on the device for the program.\\
		\subsubsection{Communications Interfaces}
			The communications interface will most likely be used to connect devices being used to run the program to\\				other devices being used to run the program. One of the uses for this will be in crowd sourcing density data of\\			the users to help effectively and intelligently create the most efficient instructions for a specific user or to\\			record and utilize the most common trends in user movements to predict behavior and increase efficiency.\\
		\subsubsection{Memory}
			Information such as maps, router positions and user identities will most likely need to be stored on a server\\ 			by the program administrators. Information gathered about the users, such as common routes, time spent using\\ 				the program, user information and any other data collected by the program should also be stored in the same\\ 				way.However, an alternative would be to store the information on the user’s device and report it to the server\\ 			when needed.\\   
		\subsubsection{Operations}
			The primary operation of the program will be directing students around the University of Pretoria’s main\\				campus, secondary operations may include crowd sourcing utilities and a rewards program. It should be possible\\			to store and retrieve information about a user’s movements in order to back trace if necessary.\\
		\subsubsection{Site Adaptation Requirements}
			The program should be available in a variety of language options.\\
		
	\subsection{Product Functions}
	\subsection{User Characteristics}
	\subsection{Constraints}
	\subsection{Assumptions and Dependencies}

\section{Specific Acquirements}

	\subsection{External Interface Requirements}
	\subsection{Functional Requirements}
	\subsection{Performance Requirements}
	\subsection{Design Constraints		\\
		1. User Movement		\\
			Because user location changes, to account for the location of the user in a building with one/ weak  \\
			Wi-Fi connection will lead to failure in updating to user location.		\\
		2. Failure connection		\\
			Since the app uses Wi-Fi to find location and navigate if the user fails to connect to Tuks Wi-Fi \\
			at that moment the app will not work regardless of whether the user uses cellular network data.		\\
		3. Location in buildings 	\\
			Finding the current location of the user and displaying it on the app inside the buildings will \\
			be restricted since we don’t have the lay-out/map of all buildings.	\\
			
	}
	\subsection{Software System Attributes\\
		1.Availability 	\\
			NavUP app is available only in the Hatfield campus and also is the user is Wi-Fi connected. \\
			This app it is only functional only if the two above requirement are met.\\
		2.Interoperability \\
			NavUP app will uses google maps as lay-out of the campus map and gps to navigate through the\\
			campus to find shortest path. \\
		3.Reliability \\
			NavUP app functionality is reliable only in the campus and when the user is connected to Tuks Wi-Fi.\\
			Therefore fi user fails to connect to the Wi-Fi they won’t be any service for the user. Just like \\
			availability, reliability depends on the requiems of the app to be met.  \\
		4.Maintainability \\
			Since this is an app NavUP can undergo updates without affecting the users systems and gives a\\
			user an option to update when the update are available.\\
		5.Supportability\\
			NavUP does tell the user what is needed for it to function by letting the user know that Tuks\\
			Wi-Fi needed or required to be at Hatfield campus. \\

	
	}
	\subsection{Other Requirements}

\clearpage
\section{Open Issues}
\subsection {GitHub Repository}
Team Maroon Repository: \url{https://github.com/josephbondjobo/COS301-Maroon-Team/tree/develop}

This repository contains:
\begin{enumerate}
\item All work done by team members.
\item \href{https://github.com/josephbondjobo/COS301-Maroon-Team/blob/develop/Intro/Contributors.md}{CONTRIBUTORS.md} file outlining which members where involved in which phases of the project.
\end{enumerate}



\newpage
\clearpage
\addcontentsline{toc}{section}{References}

\end{document}
