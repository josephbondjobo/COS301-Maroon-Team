\documentclass[a4paper,10pt]{article}

\usepackage[margin=2cm]{geometry}
\usepackage{graphicx}
\usepackage{amsmath}
\usepackage{array}
\usepackage{hyperref}
\usepackage[all]{hypcap}
\usepackage{listings}
\lstdefinestyle{TerminalStyle}{
  language=bash,
  basicstyle=\small\sffamily,
  numbers=left,
  numberstyle=\tiny,
  numbersep=3pt,
  frame=tb,
  columns=fullflexible,
  linewidth=0.9\linewidth,
  xleftmargin=0.1\linewidth
}
\lstdefinestyle{HtmlStyle}{
  language=html,
  basicstyle=\small\sffamily,
  numbers=left,
  numberstyle=\tiny,
  numbersep=3pt,
  frame=tb,
  columns=fullflexible,
  linewidth=0.9\linewidth,
  xleftmargin=0.1\linewidth
}
\lstdefinestyle{OutputStyle}{
  language=html,
  basicstyle=\small\sffamily,
  frame=tb,
  columns=fullflexible,
  linewidth=0.9\linewidth,
  xleftmargin=0.1\linewidth
}

\setlength{\parindent}{0pt}
\setlength{\parskip}{1ex plus 0.5ex minus 0.2ex}

\title{\includegraphics[width=12cm]{Eeufeeslogo.jpg} \\
       Department of Computer Science \\
       University of Pretoria \\
       \vspace{0.5cm}
       Software Engineering\\
       COS301 MiniProject \\
       \vspace{0.5cm}
       \begin{large} \textbf{Team Maroon}\\ NavUP\end{large}

\date{} 
\author{Bondjobo, Jocelyn (J) 13232852 			\\
		Mweshi, George (G)		16394713		\\
		Letsoalo, Joseph (J)	15043844		\\
		Setoaba, Phuti (P)		13032616		\\
		Trivella, Camron (C)	14070970		\\
		Coetzer, Albert (A)		15244882		\\
}

\begin{document}
\maketitle
\thispagestyle{empty}
\clearpage

\newpage
\pagenumbering{roman}
\thispagestyle{empty}
\tableofcontents
\clearpage

\newpage
\pagenumbering{arabic}

\section{Introduction}
The client Vreda Pieterse, from the University of Pretoria has requested a system called NavUp to guide students. Students and visitors to the the campus often have to find venues to attend classes, meetings and other events or to find public services such as shops, restaurants and ablution facilities. They may want to determine the optimal route from one venue to another on a regular basis, especially at the beginning of a term when classes are presented for the first time. \\

\begin{enumerate}
\item Basic functionality
	\begin{itemize}
	\item The current location of the user should be determined both outdoors and indoors.
	\item Basic navigational functions required such searching for locations, saving locations and providing directions to a location.
	\item Front end interface
	\end{itemize}
\item Surveillance
	\begin{itemize}
	\item Functionality to provide and visualise information related to pedestrian traffic based on users last/currently using the application on campus.
	\end{itemize}
\item Targeted delivery of information
	\begin{itemize}
	\item The system can automatically load/push new information to users according to their preferences and interests.
	\end{itemize}
\item Activities
	\begin{itemize}
	\item Various activities that uses location and movement of users can be integrated into the system.
	\end{itemize}
\end{enumerate}

\section{Background}
This project was commissioned based on the following needs faced by the client:
\begin{enumerate}
\item Current situation:
	\begin{itemize}
	\item The sutents and/or visitors on campus must always ask around to people to find where a venue/building is.
	\item The sutents and/or visitors on campus had to refer to the closest map and locate their position on the map and remember direction of the building they want to go.
	\end{itemize}
\item Google map
	\begin{itemize}
	\item  Google map cannot provide navigation indoors as it is limited with the GPS satellite connection. Thus, people cannot be guided to specific venues around the campus.
	\end{itemize}
\end{enumerate}
\section{Architecture requirements}
\subsection{Access channel requirements}
\subsubsection{Human Access Channels}

\subsubsection{System Access Channels}

\subsection{Quality requirements}
\subsubsection{Performance}
\begin{enumerate}
\item \textbf{Description} \\

\item \textbf{Justification} \\
\item \textbf{Requirements}
\end{enumerate}
\subsubsection{Reliability}
\begin{enumerate}
\item \textbf{Description} \\

\item \textbf{Justification} \\ 
\item \textbf{Requirements}
\end{enumerate}
\subsubsection{Scalability}
\begin{enumerate}
\item \textbf{Description} \\

\item \textbf{Justification} \\
\item \textbf{Requirements}
\end{enumerate}
\subsubsection{Security}
\begin{enumerate}
\item \textbf{Description} \\

\item \textbf{Justification} \\

\item \textbf{Requirements}

\end{enumerate}
\subsubsection{Flexibility}
\begin{enumerate}
\item \textbf{Description} \\
\item \textbf{Justification} \\

\item \textbf{Requirements}
\end{enumerate}
\subsubsection{Maintainability}
\begin{enumerate}
\item \textbf{Description} \\
\item \textbf{Justification} \\
\item \textbf{Requirements}
\end{enumerate}
\subsubsection{Auditability/Monitorability}
\begin{enumerate}
\item \textbf{Description} \\
\item \textbf{Justification} \\
\item \textbf{Requirements}
\end{enumerate}
\subsubsection{Integrability}
\begin{enumerate}
\item \textbf{Description} \\
\item \textbf{Justification} \\
\item \textbf{Requirements}
\end{enumerate}
\subsubsection{Cost}
\begin{enumerate}
\item \textbf{Description} \\
\item \textbf{Justification} \\
\item \textbf{Requirements}
\end{enumerate}
\subsubsection{Usability}
\begin{enumerate}
\item \textbf{Description} \\
\item \textbf{Justification} \\
\item \textbf{Requirements}
\end{enumerate}

\subsection{Integration requirements}
The Android application will be built to work on all Android version 4.0.  and upward devices. The mobile application on the Android device should be sensitive to the amount of data that is transmitted over the network due to the high cost of mobile internet in South Africa. To allow for minimal communication but maximum functionality, the MVC based design will be followed in the Android application design. 
\subsection{Architecture constraints}
\begin{itemize}
\item Dependency Injection (DI) will be used because it eases software testing and re-usability of software by using a design based on independent classes/components.
\item Unit Testing to automate the following tasks: compilation of Java source code, running test cases and generating project documentation.
\end{itemize}
\subsubsection{Architectural Patterns}

\clearpage
\section{Open Issues}
\subsection {GitHub Repository}
Team Maroon Repository: \url{https://github.com/josephbondjobo/COS301-Maroon-Team/tree/develop}

This repository contains:
\begin{enumerate}
\item All work done by team members.
\item \href{https://github.com/josephbondjobo/COS301-Maroon-Team/blob/develop/Intro/Contributors.md}{CONTRIBUTORS.md} file outlining which members where involved in which phases of the project.
\end{enumerate}



\newpage
\clearpage
\addcontentsline{toc}{section}{References}

\end{document}
